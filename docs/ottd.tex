\documentclass[12pt,a4paper]{report}
\usepackage[utf8x]{inputenc}
\usepackage[russian]{babel}
\usepackage{ucs}
\usepackage{amsmath}
\usepackage{amsfonts}
\usepackage{amssymb}
\usepackage{makeidx}
\usepackage{graphicx}
\usepackage[left=2.00cm, right=2.00cm, top=2.00cm, bottom=2.00cm]{geometry}

\makeatletter
\renewcommand{\@listI}{%
\topsep=10pt }
\makeatother

\begin{document}
По методам:
\begin{enumerate}
\item Метод аналогий и подобия. Для модификации алгоритма
\item Метод <<И-ИЛИ>> дерева. Для выбора структуры клеточного поля
\item Метод поиска оптимальных параметров. Для параметров игры
\item Метод коллективного блокнота. 
\item Метод активации ассоциативных связей
\item Метод экспертных оценок для оценки результата. Для общей оценки
\item Системное конструирование по Ханзену. Для игровой составляющей
\item Метод синектики. GUI, только позиционирование.
\item Метод проб и ошибок. Для подбора цветовой гаммы.
\end{enumerate}

По этапам:
\begin{enumerate}
\item Что будем симулировать? (4)
\item Определение алгоритма КА. (1)
\item Разработка игровой составляющей. (7)
\item Выбор структуры клеточного поля. (2)
\item Определение оптимальных параметров распространения. (3)
\item GUI (8)
\item Подбор цветов (9)
\item Название игры (5)
\item Общая оценка игры (6)
\end{enumerate}
\end{document}